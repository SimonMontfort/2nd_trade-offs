%% start of file `template.tex'.
%% Copyright 2006-2013 Xavier Danaux (xdanaux@gmail.com).
%
% This work may be distributed and/or modified under the
% conditions of the LaTeX Project Public License version 1.3c,
% available at http://www.latex-project.org/lppl/.


\documentclass[10.5pt,a4paper,sans]{moderncv}        % possible options include font size ('10pt', '11pt' and '12pt'), paper size ('a4paper', 'letterpaper', 'a5paper', 'legalpaper', 'executivepaper' and 'landscape') and font family ('sans' and 'roman')
\usepackage{graphbox}

% moderncv themes
\moderncvstyle{oldstyle}                           % style options are 'casual' (default), 'classic', 'oldstyle' and 'banking'
\moderncvcolor{black}                               % color options 'blue' (default), 'orange', 'green', 'red', 'purple', 'grey' and 'black'
%\renewcommand{\familydefault}{\sfdefault}         % to set the default font; use '\sfdefault' for the default sans serif font, '\rmdefault' for the default roman one, or any tex font name
\nopagenumbers{}                                  % uncomment to suppress automatic page numbering for CVs longer than one page

% character encoding
\usepackage[T1]{fontenc}
\usepackage[utf8]{inputenc}
%\usepackage{CJKutf8}                              % if you need to use CJK to typeset your resume in Chinese, Japanese or Korean


% adjust the page margins
\usepackage[scale=0.83]{geometry}
\setlength{\hintscolumnwidth}{3.3cm}                % if you want to change the width of the column with the dates
%\setlength{\makecvtitlenamewidth}{10cm}           % for the 'classic' style, if you want to force the width allocated to your name and avoid line breaks. be careful though, the length is normally calculated to avoid any overlap with your personal info; use this at your own typographical risks...


% personal data
\name{\includegraphics[scale = .05, vshift = 1.5cm]{epfl.jpg}\vspace{-.42cm}}{}
  

  
\address{\textbf{Dr. Simon Montfort}\\ EPFL ENAC IIE\\ HERUS \\ GR C1 482\\ (Batiment GR) \\ Station 2 \\1015 Lausanne \\Switzerland\\\\\\\\\\\\\\\\\\\\\\\\\\\\\\\\\\\\\\\\\\\\\\\\\\\\\\\\\\\\\\\\\\\\\ \emph{\footnotesize \underline{Article specs}\\ Words (abstract): add\\ Words (article): add\\  Figures (article): 6\\ Tables (article): 0\\ References: add}}{}{}% optional, remove / comment the line if not wanted; the ''postcode city'' and and ''country'' arguments can be omitted or provided empty             

 

%\photo[64pt][0.4pt]{picture}                       % optional, remove / comment the line if not wanted; '64pt' is the height the picture must be resized to, 0.4pt is the thickness of the frame around it (put it to 0pt for no frame) and 'picture' is the name of the picture file


% to show numerical labels in the bibliography (default is to show no labels); only useful if you make citations in your resume
%\makeatletter
%\renewcommand*{\bibliographyitemlabel}{\@biblabel{\arabic{enumiv}}}
%\makeatother
%\renewcommand*{\bibliographyitemlabel}{[\arabic{enumiv}]}% CONSIDER REPLACING THE ABOVE BY THIS

% bibliography with mutiple entries
%\usepackage{multibib}
%\newcites{book,misc}{{Books},{Others}}
%----------------------------------------------------------------------------------
%            content
%----------------------------------------------------------------------------------
\begin{document}

% recipient data
\pagestyle{plain}
\recipient{PLOS}{1875 Mission Street\\ Suite 103 \#188 San Francisco \\  CA 94103\\ United States\\ \today}
\date{\textbf{Submission Cover Letter}}

\opening{Dear Prof. Khosla, Prof. Anadon, Prof. Jaramillo, Prof. Jaramillo, dear Editors}


\makelettertitle

\rfoot{\small p. 1/1}
We are pleased to submit our manuscript entitled ``Own trade-off and synergy beliefs, not others' beliefs, drive public acceptance of energy technologies" for consideration in Environmental Research Letters.

Our paper examines how first-order beliefs (people's own perceptions of trade-offs and synergies) and second-order beliefs (expectations about what others value) shape public acceptance of energy technologies, including Alpine photovoltaics (PV), wind, sustained nuclear, and new nuclear energy. Using a population-representative survey experiment in Switzerland (N = 1,899), we combine analysis of the drivers of first- and second-order beliefs with a belief-updating experiment to test whether providing accurate information about societal preferences shifts perceptions and acceptance.

We show that entrenched first-order beliefs dominate acceptance -- this contrasts with prior research highlighting the importance of what others believe for people's own preferences. Our experimental evidence shows that when respondents with biased expectations were presented with information about societal preferences, they rarely revised their own evaluations or acceptance of energy technologies. This may be due to the high polarization around the energy technologies that we analyzed. 

Perceptions of synergies matter more than trade-offs for renewables. Respondents generally see biodiversity and landscape protection as synergistic with emission reductions, and these perceptions strongly predict acceptance of Alpine PV and wind projects. By contrast, nuclear energy acceptance is lower and less sensitive to these perceptions.

These results have broader implications for energy policy and social acceptance. They suggest that political debates may overemphasize trade-offs to advance specific agendas, but in contexts with high public awareness, preferences remain relatively stable despite misperceptions about societal views. Interventions aiming to shift acceptance may thus need to go beyond arguments about others' acceptance, leveraging participatory processes, procedural fairness, and benefit-sharing mechanisms to address locally entrenched concerns.

We believe this manuscript will be of interest to the readership of ERL, given its relevance to energy transitions, public perceptions of renewable and nuclear energy, and the interaction of social beliefs with policy support.


\\

Thank you for considering our submission. 
Sincerely,
\\
\includegraphics[scale = .04]{signature.tiff}\\
\vspace{.1cm}Dr. Simon Montfort on behalf of Dr. Jonas Schmid, Jair Campfens, Prof. Isabelle Stadelmann-Steffen, Prof. Claudia R. Binder

\end{document}


%% end of file `template.tex'.
